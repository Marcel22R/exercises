\input{../../../template.ltx}

\begin{document}

\osuetitle{1}

%\section*{Aufgabenstellung A -- mycompress}
\section*{Assignment A -- mycompress}

%Schreiben Sie ein C-Programm \osueprog{mycompress},
%das die als Argumente übergebenen Eingabedateien mittels
%eines einfachen Algorithmus komprimiert.
%Wird keine Eingabedatei angegeben, so ist von
%\osueglvar{stdin} zu lesen.
Write a C-program \osueprog{mycompress},
which reads and compresses input files with a simple algorithm.

\begin{verbatim}
    SYNOPSIS:
        mycompress [-o outfile] [file...]
\end{verbatim}

%Das Programm soll den Inhalt der übergebenenen Eingabedateien
%nacheinander auslesen, komprimieren, und die komprimierte Form in ein
%File mit dem durch die Option \verb|-o| spezifizierten Namen
%\osuefilename{outfile} ausgeben.
%Werden keine Eingabedateien angegeben, so ist von \osueglvar{stdin} zu
%lesen.
%Fehlt die Option \verb|-o|, wird auf \osueglvar{stdout} ausgegeben.
The program shall read the content of any files given as positional arguments
one after the other, compress it,
and write the compressed content to an output file \osuefilename{outfile}
given by the option \verb|-o|.
If no input files are given, the program reads from \osueglvar{stdin}.
If no output file is given, the program writes to \osueglvar{stdout}.

Your program must accept lines of any length.

%Die Komprimierung soll so erfolgen, dass die Zeichen durch
%\verb|Zeichen + Anzahl| ersetzt werden -- z.B.: \verb|aaa| durch
%\verb|a3| und \verb|b| durch \verb|b1|.
%Das Newline ist auch ein Zeichen und soll ebenfalls komprimiert werden
%(siehe \verb|'1'| zu Beginn einer Zeile im folgenden Beispiel).
The input is compressed by substituting subsequent identical characters
by only one occurence of the character followed by the number of characters.
For example, if you encounter the sequence \verb|aaa|,
then it is replaced by \verb|a3|.
If a character appears only once in sequence,
then it is printed followed by a 1.
Therefore the sequence \verb|b| is converted to \verb|b1|.
The newline character is also processed like any other character
(which in the output results in numbers being printed at the start of each line,
except for the first line;
those are actually the number of subsequent newline characters).

%Geben Sie die Anzahl der gelesenen Zeichen, die Anzahl der geschriebenen
%Zeichen und die daraus resultierende Komprimierungsrate
%auf die Standardfehlerausgabe \osueglvar{stderr} aus.
%(Sie werden feststellen, dass die Komprimierung nur bei vielen gleichen
%Zeichen effizient ist.)
Print the number of read characters, the number of written characters
and the resulting compression ratio to \osueglvar{stderr}.

%Definieren Sie für die maximale Anzahl an Zeichen in einer
%Zeile eine Konstante, wobei Sie annehmen dürfen, dass keine
%Zeile mehr Zeichen enthält.

%\subsection*{Testen}
\subsection*{Testing}
%Testen Sie Ihr Programm mit verschiedenen Eingaben; z.B.\ soll
%die Eingabe
Test your program with various inputs, such as:

\begin{osuefmtcode}
aaabbbbbc
dddddddde

ggghhhhha
eeeeefffg
\end{osuefmtcode}

which shall write following output to the output file or to \osueglvar{stdout}:

\begin{osuefmtcode}
a3b5c1
1d8e1
2g3h5a1
1e5f3g1
1
\end{osuefmtcode}

and following output to \osueglvar{stderr}:

\begin{osuefmtcode}
Read:      41 characters
Written:   30 characters
Compression ratio: 73.2\%
\end{osuefmtcode}

\osueguidelinesone

\end{document}

