\input{../../../template.ltx}

\begin{document}

\osuetitle{1}

%\section*{Aufgabenstellung A -- myexpand}
\section*{Assignment A -- myexpand}

%Implementieren Sie eine vereinfachte Variante des Unix-Kommandos
%\osueprog{expand}.
Implement a variation of the Unix-command \osueprog{expand}.
Write a C-program \osueprog{myexpand},
which reads in several files and replaces tabs with spaces.

\begin{verbatim}
    SYNOPSIS:
        myexpand [-t tabstop] [-o outfile] [file...]
\end{verbatim}

%Das Programm \osueprog{myexpand} soll die als Argumente angegebenen Dateien
%lesen. Ist keine Datei angegeben, soll von \osueglvar{stdin} gelesen werden.
%Dabei werden auftretende Tabs durch Leerzeichen ersetzt. Die Ausgabe
%soll auf \osueglvar{stdout} erfolgen. Der optionale Parameter \osuearg{tabstop}
%gibt an, an welchen Positionen die Tabs enden sollen (fehlt dieser,
%ist der Wert 8 anzunehmen).

%\subsection*{Anleitung}
%Lesen Sie die Dateien zeichenweise ein und überprüfen Sie den
%ASCII-Code des eingelesenen Zeichens. Handelt es sich um ein Tab
%(\verb|\t|), berechnen Sie die Position \verb|p| des folgenden
%Zeichens als nächstes Vielfaches von \verb|tabstop| größer der
%aktuellen Position plus 1:

%\verb|p = tabstop * ((x / tabstop) + 1)|

%wobei \verb|x| die Position des Tabs in der aktuellen Zeile und
%\verb|/| eine ganzzahlige Division (mit Abschneiden der
%Nachkommastellen) beschreibt.

%\verb|\t| steht für ein Tab-Zeichen und ist eine \emph{Escape
  %Sequence}\footnote{\url{https://en.wikipedia.org/wiki/Escape_sequences_in_C}},
%d.h. \verb|\t| wird durch ein horizontales Tab ersetzt.

The program shall read each file given as positional argument
(or \osueglvar{stdin} if there are no positional arguments)
line by line and search for tab characters (\verb|\t|).

Your program must accept lines of any length.

Each of these tabs is replaced with a number of space characters,
such that the next character
is placed at the next multiple of the tabstop distance within the line.
This position can be calculated as follows:

\verb|p = tabstop * ((x / tabstop) + 1)|

where \verb|x| is the position of the tab character
(after expanding any previous tabs on that line)
and \verb|/| is the standard integer division (retaining only the quotient).

The tabstop distance is 8 by default,
but this can be overridden with an arbitrary positive integer
using the option \osuearg{-t}.

If the option \osuearg{-o} is given,
the output is written to the specified file (\verb|outfile|).
Otherwise, the output is written to \osueglvar{stdout}.

%\subsection*{Testen}
\subsection*{Testing}

%Testen Sie Ihr Programm mit mehreren Eingabedateien. Erstellen Sie z.B. eine
%Testdatei \osuefilename{t1} mit folgendem Inhalt:
Test your program with various inputs,
such as a file \osuefilename{t1.txt} with following content
(where \verb|\t| represents the ASCII tab character,
entered by hitting the tabulator key):

\begin{osuefmtcode}
    1234567890
    123\bslash{}t567890
    \bslash{}t\bslash{}t90
\end{osuefmtcode}

%Rufen Sie Ihr Programm dann wie folgt auf:
Executing your program should give an output similar to the following:

\begin{osuefmtcode}
    $ \osueinput{./myexpand t1.txt}
    1234567890
    123     567890
                    90

    $ \osueinput{./myexpand -t 4 -o out.txt t1.txt}
    $ \osueinput{cat out.txt}
    1234567890
    123 567890
            90
\end{osuefmtcode}

\osueguidelinesone

\end{document}
