\input{../../../template.ltx}

\begin{document}

\osuetitle{1}

%\section*{Aufgabenstellung A -- mydiff}
\section*{Assignment A -- mydiff}

%Implementieren Sie eine Abwandlung des Unix-Kommandos \osueprog{diff}.
%Schreiben Sie zu diesem Zweck ein C-Programm
%\osueprog{mydiff}, das zwei übergebene Dateien zeilenweise
%miteinander vergleicht und, falls sich zwei Zeilen voneinander
%unterscheiden, die Zeilennummer und die Anzahl der
%verschiedenen Zeichen ausgibt.
Implement a variation of the Unix-command \osueprog{diff}.
Write a C-program \osueprog{mydiff},
which reads in two files und compares them.
If two lines differ, then the line number
and the number of differing characters is printed.

\begin{verbatim}
    SYNOPSIS:
        mydiff [-i] [-o outfile] file1 file2
\end{verbatim}

%\subsection*{Anleitung}
%Das Programm soll jede Datei zeilenweise einlesen und die
%gelesenen Zeichen miteinander vergleichen. Sind die Zeilen
%ungleich lang, so soll nur bis zur Länge der kürzeren Zeile
%verglichen werden -- z.B.: \verb+Haus\n+ und \verb+Haustor\n+
%sind als gleich zu behandeln. Falls eine Datei mehr Zeilen enthält
%als die andere, sollen die restlichen Zeilen ebenfalls ignoriert
%werden. Pro Zeile soll gezählt werden, wieviele Zeichen an der
%gleichen Position nicht übereinstimmen. Falls die Option \verb|-i|
%angegeben ist, soll Groß-/Kleinschreibung nicht als Unterschied
%gezählt werden.
The program shall read each file line by line
und compare the characters.
If two lines have different length,
then the comparison shall stop upon reaching the end of the shorter line.
Therefore, the lines \verb+abc\n+ und \verb+abcdef\n+
shall be treated as being identical.

%Definieren Sie für die maximale Anzahl an Zeichen in einer
%Zeile eine Konstante, wobei Sie annehmen dürfen, dass keine
%Zeile mehr Zeichen enthält.
Your program must accept lines of any length.

If the option \osuearg{-o} is given,
the output is written to the specified file (\verb|outfile|).
Otherwise, the output is written to \osueglvar{stdout}.

If the option \osuearg{-i} is given,
the program shall not differentiate between lower and upper case letters,
i.e. the comparison of the two lines shall be case insensitive.

\paragraph*{Hint:}
Take a look at the functions \verb|strncmp(3)| and \verb|strncasecmp(3)|
for comparing two lines.

%\subsection*{Testen}
\subsection*{Testing}

%Erstellen Sie eine Testdatei \osuefilename{difftest1.txt} mit folgendem Inhalt:
Test your program with various inputs,
such as a file \osuefilename{difftest1.txt} with following content:

\begin{osuefmtcode}
      abc
      operating
      abcdefg
\end{osuefmtcode}

%und eine zweite Testdatei \osuefilename{difftest2.txt} mit folgendem Inhalt:
and a file \osuefilename{difftest2.txt} with following content:

\begin{osuefmtcode}
      abcdefg
      Operating Systems
      ahciejg
      abcdefg
\end{osuefmtcode}

%Rufen Sie Ihr Programm dann mit folgenden Argumenten auf:
Executing your program should give an output similar to the following:

\begin{osuefmtcode}
    $ \osueinput{./mydiff difftest1.txt difftest2.txt}
    Line: 2, characters: 1
    Line: 3, characters: 3
    $ \osueinput{./mydiff -i difftest1.txt difftest2.txt}
    Line: 3, characters: 3
    $ \osueinput{./mydiff -o example.out difftest1.txt difftest2.txt}
    $ \osueinput{cat example.out}
    Line: 2, characters: 1
    Line: 3, characters: 3
\end{osuefmtcode}

\osueguidelinesone

\end{document}
