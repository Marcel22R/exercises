\input{../../../template.ltx}

\begin{document}

\osuetitle{1}

\section*{Assignment A -- mygrep}

Implement a reduced variation of the Unix-command \osueprog{grep}.
Write a C-program \osueprog{mygrep},
which reads in several files and prints all lines containing a keyword.

\begin{verbatim}
    SYNOPSIS
        mygrep [-i] [-o outfile] keyword [file...]
\end{verbatim}

The program \osueprog{mygrep} shall read files line by line
and for each line check whether it contains the search term \osuearg{keyword}.
The line is printed if it contains \osuearg{keyword},
otherwise it is not printed.

Your program must accept lines of any length.

If one or multiple input files are specified
(given as positional arguments after \osuearg{keyword}),
then \osueprog{mygrep} shall read each of them in the order they are given.
If no input file is specified, the program reads from \osueglvar{stdin}.

If the option \osuearg{-o} is given,
the output is written to the specified file (\verb|outfile|).
Otherwise, the output is written to \osueglvar{stdout}.

If the option \osuearg{-i} is given,
the program shall not differentiate between lower and upper case letters,
i.e. the search for the keyword in a line is case insensitive.

\paragraph*{Hint:}
Take a look at the functions \verb|strstr(3)| and \verb|strcasestr(3)|.

\subsection*{Testing}

Test your program with various inputs, such as:

\begin{osuefmtcode}
      $ \osueinput{./mygrep rat}
      \osueinput{Trap a rat!}
      Trap a rat!
      \osueinput{Never odd or even}
      \osueinput{Rating System}
      \osueinput{Operating Systems}
      Operating Systems

      $ \osueinput{./mygrep -i rat}
      \osueinput{Rating System}
      Rating System
      \osueinput{Operating Systems}
      Operating Systems

      $ \osueinput{cat example.in}
      Operating Systems
      Rating System
      $ \osueinput{./mygrep -o example.out rat example.in}
      $ \osueinput{cat example.out}
      Operating Systems
\end{osuefmtcode}

\osueguidelinesone

\end{document}

