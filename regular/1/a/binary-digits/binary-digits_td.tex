\input{../../../template.ltx}

\begin{document}

\osuetitle{1}

\section*{Assignment A -- binary-digits}

Write a C-program that visualizes data in it's binary representation.
In other words by printing \osueglvar{1}s and \osueglvar{0}s.

\begin{verbatim}

SYNOPSIS:
	binary-digits [-d DELAY] [-o OUTPUTFILE] [FILE]...

\end{verbatim}


The program shall read each file given as positional argument
(or \osuefilename{stdin} if there are no positional arguments)
and print it byte per byte to \osuefilename{stdout} or a specified output file in \textbf{big endian byte order}.
\footnote{
	\url{https://en.wikipedia.org/wiki/Endianness}\\
}

Your program must work on a stream of characters.
I.e. not wait until all input is available, but process
the data immediately.

The program must run until either all files specified as
positional arguments are processed, there is no more input
from \osuefilename{stdin} (i.e. \osueglvar{EOF} is reached)
or the program receives some signal demanding termination,
which ever comes first.
No special signal handling is necessary.

For example a call to program with a file as an argument
only containing the ASCII characters \verb|a|, \verb|b|
and a new-line character (\verb|\n|), should produce this
output:

\verb|011000010110001000001010|

\verb|3 * 8| digits, because every character is represented
by \verb|8| bits.

If the option \osuearg{-o} is given,
the output is written to the specified file \osuefilename{outfile}.
Otherwise, the output is written to \osuefilename{stdout}.

If the option \osuearg{-d} is given,
the program should wait the specified amount of seconds after each
digit written.
The delay may be specified as an ASCII represented decimal floating point number, e.g. 0.5.

\subsection*{Hints}
\begin{itemize}
	\item The functions \osuefunc{nanosleep(2)} and \osuefunc{strtod(3)}
	might be helpful for implementing the delay part.
	
	\item \osuefunc{fgetc(3)} and \osuefunc{fputc(3)} can be used to operate
	on character streams.

	\item It might be necessary to \osuefunc{fflush(3)} a stream in order
	to generate steady output.

\end{itemize}

%\pagebreak

\subsection*{Testing}

Test your program with various inputs,
such as a file \osuefilename{in1.txt} containing \verb|"OSUE\n"|.

Executing your program should produce:

\begin{osuefmtcode}
	
$ \osueinput{./binary-digits in1.txt}
0100111101010011010101010100010100001010

\end{osuefmtcode}

Or you can let the programm read from \osuefilename{stdin}
interactively, by just calling:

\begin{osuefmtcode}

	$ \osueinput{./binary-digits}

\end{osuefmtcode}

And then entering arbitrary letters on your keyboard
this way you can see the representation for anything you want.
Note that the shell input is buffered and you will only see
a result every time a new-line character (\verb|\n|) is sent
by pressing return. End the input with Ctrl+D.

A more complex invocation could be:

\begin{osuefmtcode}
	
	$ \osueinput{./binary-digits -o outfile.txt -d 0.5 in1.txt in2.txt}
\end{osuefmtcode}

This will store the contents of \osuefilename{in1.txt} and
\osuefilename{in2.txt} expanded to \verb|1| and \verb|0| characters in
\osuefilename{outfile.txt}, but wait \verb|0.5| seconds after writing
each character.

What you could also do is read random input from \osueprog{/dev/urandom}
and create an endless stream of text.

\begin{osuefmtcode}

	$ \osueinput{./binary-digits -d 0.05 /dev/urandom}

\end{osuefmtcode}


\osueguidelinesone

\end{document}
