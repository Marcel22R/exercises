\input{../../template.ltx}

\begin{document}

\osuetitle{2}

\section*{Aufgabenstellung -- forksort}
%Schreiben Sie ein Programm, das Eingaben sortiert.
Implement an algorithm which sorts lines alphabetically.
\begin{verbatim}
    SYNOPSIS
        forksort
\end{verbatim}

%\subsection*{Anleitung}
%Das Programm soll die Daten von \osueglvar{stdin} lesen. Die erste Zeile muss
%eine Zahl größer Null sein, die die Anzahl der zu sortierenden Strings angibt.
%Die darauffolgenden Zeilen enthalten die Strings selbst; Sie können die
%Strings auf je 62 echte Zeichen begrenzen. (Vergessen Sie nicht dafür eine
%Konstante zu definieren!)
\subsection*{Instructions}
The program takes multiple lines as input
and sorts them by using a recursive variant of merge sort\footnote{\url{https://en.wikipedia.org/wiki/Merge_sort}}.
The input is read from \texttt{stdin}
and ends when an EOF (End Of File) is encountered.

Your program must accept any number of lines.

The program sorts the lines recursively,
i.e. by calling itself:

%\subsection*{Arbeitsweise}
%Forksort funktioniert wie eine rekursive Merge-Sort-Variante\footnote{Siehe insb. die graphisch aufbereitete Beispielausf{\"u}hrung auf: http://de.wikipedia.org/wiki/Mergesort}.
\begin{enumerate}
%\item Lesen Sie die Anzahl der Strings ein.
%\item Ist die Anzahl gleich Eins, dann geben Sie das "`sortierte"' Ergebnis
  %aus (d.h. den Eingabe-String selbst).
\item If the input consist of only 1 line,
then write it to \texttt{stdout} and exit.
%\item Ist die Anzahl größer Eins, erstellen Sie zuerst vier Unnamed-Pipes (zwei
  %je Kind) und erzeugen dann zwei Kindprozesse (nicht Kind- und Enkelkind) mittels
  %\osuefunc{fork(2)}. Verwenden Sie \osuefunc{dup2(2)}
  %und \osuefunc{execlp(3)} um \osueglvar{stdin}/\osueglvar{stdout} der neuen
  %Prozesse auf eigene Streams umzuleiten und danach Forksort rekursiv
  %aufzurufen.
%\item Schreiben Sie je die Hälfte der Strings auf \osueglvar{stdin} der
  %beiden Kinder im zuvor spezifiziertem Eingabeformat (d.h., Anzahl der Strings,
  %Newline, die Eingabestrings zeilenweise).
  %Ist die Anzahl der Strings ungerade, dann wählen Sie
  %ein beliebiges Kind aus, das einen String mehr bekommt.
  %%Vergessen Sie nicht
  %%auch die Anzahl der Strings an das jeweilige Kind zu übermitteln.
\item Otherwise the input consist of $n>1$ lines.
Split them into two parts, with each part consisting of $n/2$ lines.
If $n$ is odd, one of the parts will have one line more than the other.
\item Using \osuefunc{fork(2)} and \osuefunc{execlp(3)},
recursively execute this program in two child processes,
one for each of the two parts.
Use two unnamed pipes per child
to redirect \osueglvar{stdin} and \osueglvar{stdout}
(see \osuefunc{pipe(2)} and \osuefunc{dup2(3)}).
Write the first part to \osueglvar{stdin} of one child
and the second part to \osueglvar{stdin} of the other child.
Read the respective sorted lines from each child's \osueglvar{stdout}.
The two child processes must run simultaneously!

\item Use \osuefunc{wait(2)} or \osuefunc{waitpid(2)}
to read the exit status of the children.
Terminate the program with exit status \verb|EXIT_ERROR|
if the exit status of any of the two child processes is not \verb|EXIT_SUCCESS|.

%\item Lesen Sie nun die Ausgabe der Kinder zeilenweise ein, und
  %geben Sie diese aber nun ''verschmolzen'', d.h. sortiert zeilenweise aus
  %(dieser Vorgang nennt sich \emph{Merge}).
  %Beachten Sie, dass die Ausgabe der Kinder bereits fuer
  %sich sortiert ist.  Sie können also
  %solange die Strings des ersten Kindes ausgeben, bis der kleinste String des
  %zweiten Kindes kleiner als der kleinste, noch nicht ausgegebene, String des
  %ersten Kindes ist. Danach lesen Sie vom zweiten Kind, bis der String des
  %ersten Kindes wiederum kleiner ist. Wiederholen Sie diesen Vorgang solange,
  %bis alle Strings der Kinder abgearbeitet sind.
  %Der Merge-Vorgang muss einen linearen Aufwand haben.
\item Merge the sorted parts from the two child processes
and write them to \osueglvar{stdout}.
At each step, compare the next line of both parts
and write the smaller one to \osueglvar{stdout},
such that the lines are written in alphabetical order.

%\item Verwenden Sie \osuefunc{wait(2)} oder \osuefunc{waitpid(2)}, um den
  %Exit-Status der beiden Kinder zu lesen.
\end{enumerate}

The algorithm should be case sensitive!

%\subsection*{Hinweise}
%Achten Sie auf korrekte Terminierungsbedingungen von forksort um ''Endlos-Rekursionen'' zu vermeiden\footnote{http://en.wikipedia.org/wiki/Fork\_bomb}.
%Dazu sind zwei Regeln zu beachten:
%\begin{enumerate}
%\item Forken Sie nur, wenn die Anzahl der zu sortierenden Strings größer eins ist.
%\item Die an das jeweilige Kind übergebene Anzahl muss stets kleiner sein als
  %die des Elternprozesses.
%\end{enumerate}
%Beginnen Sie am besten, indem Sie eine Variante schreiben, die nur einen String
%sortieren kann. Erweitern Sie nun Ihr Programm für zwei Strings, indem Sie den beschriebenen
%Fork-Vorgang implementieren und jeweils einen String an jedes der beiden Kinder schreiben. Diese
%können ja schon einen einzelnen String sortieren. Ein einziger Aufruf von
%\osuefunc{strcmp(3)} als Merge-Implementation gen{\"u}gt um zu entscheiden welche
%Ausgabe der Kinder zuerst vom Elternprozess ausgegeben werden soll. Wenn
%das funktioniert, können Sie Ihre Merge-Implementierung verallgemeinern, um auch
%mehr als zwei Strings sortieren zu können.

%Um Fehlermeldungen und Debug-Messages auszugeben, verwenden Sie stets
%\osueglvar{stderr}, da die \osueglvar{stdout} in den meisten Fällen umgeleitet ist.

%Zum Testen der Merge-Implementierung können Sie mit \osuefunc{execlp(3)} statt
%Forksort auch \osuecmd{sort(1)} aufrufen. Für die endgültige Abgabe ist diese
%Vorgehensweise nicht gültig, zum Testen aber durchaus sinnvoll.

\subsection*{Hints}

\begin{itemize}
\item In order to avoid endless recursion\footnote{\url{http://en.wikipedia.org/wiki/Fork\_bomb}},
fork only if the input number is greater than 1.
\item To output error messages and debug messages, always use
\osueglvar{stderr} because \osueglvar{stdout} is redirected in most cases.
\end{itemize}

%\subsection*{Beispiele}
\subsection*{Examples}
\begin{verbatim}
$ cat 1.txt
Heinrich
Anton
Theodor
Dora
Hugo
$ ./forksort < 1.txt
Anton
Dora
Heinrich
Hugo
Theodor
\end{verbatim}

\osueguidelinestwo

\end{document}
