\input{../../template.ltx}
\usepackage{amsmath}
\usepackage{multicol}
\begin{document}

\osuetitle{2}

\section*{Assignment -- Fork Fourier Transform}
Implement the Cooley-Tukey Fast Fourier Transform\footnote{\url{https://en.wikipedia.org/wiki/Cooley-Tukey\_FFT\_algorithm}} algorithm.
\begin{verbatim}
    SYNOPSIS
        forkFFT
\end{verbatim}

\subsection*{Instructions}
The input of the program are floating point values, which should be read from \osueglvar{stdin}.
Subsequent values are separated by a newline character.
The sequence ends when an EOF (End Of File) is encountered.

Your program must accept any number of values.
Terminate the program with exit status \verb|EXIT_FAILURE|
if an invalid input is encountered.

The program computes the Fourier Transform of its input values recursively,
i.e. by calling itself:
\begin{enumerate}
\item If the array consists of only 1 number, write that number to \osueglvar{stdout}
and exit with exit status \verb|EXIT_SUCCESS|.
\item Otherwise the array consists of $n>1$ numbers.
Split it into two parts with $n/2$ numbers each,
such that one part contains all the numbers at even indices
and the second part contains all the numbers at odd indices:
\[
P_E=\{A[0],\;A[2],\;A[4],\;\dots\}
\]
\[
P_O=\{A[1],\;A[3],\;A[5],\;\dots\}
\]
Terminate the program with exit status \verb|EXIT_FAILURE|
if the length of the array is not even.

\item Using \osuefunc{fork(2)} and \osuefunc{execlp(3)},
recursively execute this program in two child processes,
one for each of the two parts.
Use two unnamed pipes per child
to redirect \osueglvar{stdin} and \osueglvar{stdout}
(see \osuefunc{pipe(2)} and \osuefunc{dup2(3)}).
Write $P_E$ to \osueglvar{stdin} of one child
and $P_O$ to \osueglvar{stdin} of the other child.
Read the respective results from each child's \osueglvar{stdout}.
The two child processes must run simultaneously!

\item Use \osuefunc{wait(2)} or \osuefunc{waitpid(2)}
to read the exit status of the children.
Terminate the program with exit status \verb|EXIT_FAILURE|
if the exit status of any of the two child processes is not \verb|EXIT_SUCCESS|.

\item Let $R_E$ be the result of the even part $P_E$
and $R_O$ be the result of the odd part $P_O$.
The result $R$ of the Fourier Transform of the entire array can now be computed
by applying the ``butterfly'' operation to the two results:
\[
\forall k\;|\;0\leq k<n/2\;:\quad R[k]=R_E[k]+e^{-\frac{2\pi i}{n}\cdot k}\cdot R_O[k]
\quad\text{and}\quad R[k+n/2]=R_E[k]-e^{-\frac{2\pi i}{n}\cdot k}\cdot R_O[k]
\]
which is identical to:
\[
\forall k\;|\;0\leq k<n/2\;:\quad
\renewcommand{\arraystretch}{1.8}
\begin{array}{r@{\;=\;}l}
 R[k]     & R_E[k]+\left(\text{cos}(-\frac{2\pi}{n}\cdot k)+i\cdot\text{sin}(-\frac{2\pi}{n}\cdot k)\right)\cdot R_O[k]\\
R[k+n/2] & R_E[k]-\left(\text{cos}(-\frac{2\pi}{n}\cdot k)+i\cdot\text{sin}(-\frac{2\pi}{n}\cdot k)\right)\cdot R_O[k]
\end{array}
\]
You may approximate the value of $\pi$ with $3.141592654$.
\item The resulting array is printed to \osueglvar{stdout}, one value per line.
Since the resulting numbers are complex, the program prints two values per line:
the real and the imaginary part of each value, separated by a whitespace
(likewise $R_E$ and $R_O$ are imaginary,
thus when reading them from the child processes,
two values per line must be parsed).
You may append \verb|*i| or similar to the second value
to indicate that it is the imaginary part (see examples).
Terminate the program with exit status \verb|EXIT_SUCCESS|.

\end{enumerate}

Throughout the program,
it is sufficient to use single precision floating point numbers.

\subsection*{Hints}
\begin{enumerate}
\item Use \osuefunc{strtof(3)} to parse floating point values.
The \verb|endptr| argument can be helpful for reading two numbers per line
when reading from the child processes.
\item Use the header \osueglvar{math.h} for the calculation of the sine and cosine.
Add \verb|-lm| to the linker options to use these functions.
\item When dealing with complex numbers,
you may find following property useful to implement multiplication:
$(a+i\cdot b)(c+i\cdot d)=a\cdot c-b\cdot d+i\cdot(a\cdot d+b\cdot c)$
Addition should be straight-forward to implement
and no other operation with complex numbers is required.
\item In order to avoid endless recursion\footnote{\url{http://en.wikipedia.org/wiki/Fork\_bomb}},
fork only if the input number is greater than 1.
\item To output error messages and debug messages, always use
\osueglvar{stderr} because \osueglvar{stdout} is redirected in most cases.
\item Due to the rejection of sequences of odd length by each (child-)process,
the algorithm has the property of only correctly processing sequences
where the length is a power of 2.
\end{enumerate}

\subsection*{Examples}
\begin{multicols}{2}
\paragraph{Constant signal:}
\begin{verbatim}
$ cat constant.txt
1.0
1.0
$ ./forkFFT < constant.txt
2.0 0.0*i
0.0 0.0*i












\end{verbatim}

\paragraph{Sinus wave:}
\begin{verbatim}
$ cat sine.txt
0.000000
0.707107
1.000000
0.707107
0.000000
-0.707107
-1.000000
-0.707107
$ ./forkFFT < sine.txt
0.000000 0.000000*i
0.000000 -4.000001*i
0.000000 0.000000*i
0.000000 -0.000001*i
0.000000 0.000000*i
-0.000000 0.000001*i
0.000000 0.000000*i
-0.000000 4.000001*i
\end{verbatim}
\end{multicols}

Small deviations of the resulting values,
which are a consequence of the limited precision of floating point numbers,
are not relevant and not considered to be an error.

\newpage
\subsection*{Bonus exercise, 5 points}
Print the parent and child relations in form of a tree to stdout. Print all children which are forked from the parent.
The tree should be readable at least to a depth of three.
For every node, the intermediate result should be printed.\newline
Depending on how often you call fork the wider the tree becomes. A simple tree example which searches for the maximum number in a set is shown below.

\begin{verbatim}
10
                            MAX(10,8,3,5,2,1,6,7)
                              /              \		
                         MAX(10,8,3,5)    MAX(2,1,6,7)
                           /        \        /      \
                      MAX(10,8)  MAX(3,5)  MAX(2,1) MAX(6,7)
                       /    \      /   \     /   \    /   \
                      10    8     3     5   2     1   6    7 
\end{verbatim}
\paragraph{Instructions on how to print the tree}
\begin{itemize}
	\item leaf node: \\
	A leaf node should print the substep executed by it to stdout with a terminating newline.
	\item inner node: 
	\begin{itemize}
		\item To get the necessary identation use several blank characters. Think about a good way to find the right number of blank characters. For example you could use precalculated values or calculate the number from the first line you read from the children.
		\item Calculate the intermediate result and print this and the executed operation to stdout.
		\item Slash and backslash, which represent the branches of the tree, are printed to stdout.
		\item Read the output from the children line by line via a pipe. This means read the first line from the first child, then the first line from the second child and so on. Remove the newline characters. Line up the results and then print it with a terminating newline to stdout. Do this for each line returned by the child.
	\end{itemize}
\end{itemize}

\osueguidelinestwo

\end{document}
