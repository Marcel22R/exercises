\input{../../template.ltx}
\usepackage{amsmath}
\begin{document}

\osuetitle{2}

\section*{ Assignment -- Integer Multiplication}
Implement an algorithm for the efficient multiplication of large integers. 
\begin{verbatim}
    SYNOPSIS
        intmul
\end{verbatim}

\subsection*{Instructions}
The program takes two hexadecimal integers $A$ and $B$ with an equal number of digits as input, multiplies them and prints the result. The number of digits might not be fixed!
\subsection*{Examples}
\begin{verbatim}
$ ./intmul
3
3
\end{verbatim}
Output: 
\begin{verbatim}
9
\end{verbatim}
\begin{verbatim}
$ ./intmul
1A
B3
\end{verbatim}
Output: 
\begin{verbatim}
122e
\end{verbatim}
\subsection*{Approach to work}
The input is read from \texttt{stdin}, the first line is the integer $A$ and the second line is the integer $B$.

The multiplication is done by using a divide and conquer algorithm:
\begin{enumerate}
\item If $A$ and $B$ both consist of only 1 hexadecimal digit, then multiply them,
write the result to \texttt{stdout} and exit.
\item Otherwise $A$ and $B$ both consist of $n>1$ digits. Split them both into two parts each,
with each part consisting of $n/2$ digits:
\renewcommand{\arraystretch}{1.3}
\[
A:\quad\quad\begin{array}{|c|c|}\hline \quad A_h \quad & \quad A_l \quad \\\hline\end{array}\quad\quad A=A_h\cdot 16^{n/2}+A_l
\]
\[
B:\quad\quad\begin{array}{|c|c|}\hline \quad B_h \quad & \quad B_l \quad \\\hline\end{array}\quad\quad B=B_h\cdot 16^{n/2}+B_l
\]
\item Using \osuefunc{fork(2)} and \osuefunc{execlp(3)}, recursively execute this program four times for each of the
multiplications $A_h\cdot B_h$, $A_h\cdot B_l$, $A_l\cdot B_h$ and  $A_l\cdot B_l$.
\item  Use unamed pipes and \osuefunc{dup2(2)} to redirect \osueglvar{stdin}/\osueglvar{stdout} of the child.
\item The result of the multiplication $A\cdot B$ can now be calculated as follows:
\[
A\cdot B=A_h\cdot B_h\cdot 16^n + A_h\cdot B_l\cdot 16^{n/2} + A_l\cdot B_h\cdot 16^{n/2} + A_l\cdot B_l
\]
Find a clever way to write the result of this operation to \texttt{stdout},
even if the numbers are too large for the C data types.

\item Use \osuefunc{wait(2)} or \osuefunc{waitpid(2)} to read the exit status of the children.

\end{enumerate}

\subsection*{Hints}

Pay attention to correct termination conditions of intmul to avoid endless recursions \footnote{http://en.wikipedia.org/wiki/Fork\_bomb}.
Fork only if the number of digits is greater than 1.


\begin{itemize}
	\item Use malloc to store the hex values
	\item Start with multiplying two hex values with only one digit and then expand it to more digits. 
\end{itemize}
To output error messages and debug messages, always use
\osueglvar{stderr} because \osueglvar{stdout} is redirected in most cases.

\osueguidelinestwo

\end{document}
