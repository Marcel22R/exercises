\input{../../template.ltx}
\begin{document}

\osuetitle{3}

\section*{Aufgabenstellung -- auth}

In dieser Aufgabe sollen Sie eine passwort-geschützte Datenbank entwickeln. Die
Implementierung soll aus zwei Programmen bestehen: einem Server, der die
Datenbank verwaltet und Anfragen über deren Inhalt bearbeitet, und einem
Client, mit welchem der Benutzer Informationen der Datenbank vom Server setzen
oder abfragen kann. Die Kommunikation zwischen den Prozessen soll mittels
Shared Memory realisiert werden und die Synchronisation über Semaphore
erfolgen.

Der Server speichert folgende Benutzerdaten - Name des Benutzers
(\texttt{username}), Passwort (\texttt{password}) und ein Geheimnis
(\texttt{secret}) - in einer Datenbank. Der Client soll dem Benutzer ein
Interface bieten, mit dem der Benutzer diese Informationen vom Server setzen
und abfragen kann.

\subsection*{Anleitung}

Die Kommunikation zwischen den Clients und dem Server soll mittels einem
einzigen Shared Memory Object erfolgen (\textbf{nicht} einem pro Client). Es
darf auch nicht die gesamte Datenbank im Shared Memory geladen sein,
sondern nur die Information, die der Server mit einem einzigen Client
austauscht. Allerdings muss eine beliebige Anzahl von Clients gleichzeitig und
unabhängig voneinander mit dem Server kommunizieren können.

Implementieren Sie ein geeinetes Protokoll, mit welchem die Clients und der
Server über das Shared Memory Object kommunizieren. Überlegen Sie sich dafür
eine Struktur, mit welcher Anfragen und Antworten im Shared Memory gespeichert
werden. Legen Sie so viele Semaphore an, wie für die Synchronisierung der
Kommunikation zwischen dem Server und den Clients benötigt werden. Achten Sie
darauf Deadlocks zu vermeiden!

Server und Client sollen die Freigabe des Shared Memory Objects und der
Semaphore koordinieren. Spätestens wenn der Server und alle Clients terminiert
haben, müssen alle Semaphore und Shared Memory Objects freigegeben sein.
Sollte der Server vor einigen der Clients terminieren, dann müssen auch alle
verbleibenden Clients unverzüglich terminieren.

\subsubsection*{Server}
\begin{verbatim}
USAGE: auth-server [-l database]
\end{verbatim}

Der Server legt zu Beginn die benötigten Resourcen an. Anschließend lädt der
Server die angegebene Datenbank in einer geeigneten Datenstruktur in den
Speicher. Wird diese Option beim Start nicht angegeben, so verwenden Sie bitte
eine leere Datenbank als Ausgangspunkt. Die Datenbank dient dem Server im
Weiteren als Informationsquelle für das Bearbeiten der Client-Anfragen. Gibt es
ein Problem beim Einlesen der Datenbank, so beenden Sie den Server mit einer
Fehlermeldung.

Nach der Initialisierung, bearbeitet der Server Anfragen von Clients. Der
Server legt den vom Client gewünschten Eintrag an oder sucht den entsprechenden
Eintrag in der Datenbank und sendet dem Client eine Antwort mit der gewünschten
Information.

Der Server soll bei \textit{jeder Terminierung} seine Datenbank in eine Datei
names \texttt{auth-server.db.csv} schreiben. Falls die Datei bereits existiert,
soll diese einfach überschrieben werden.

\subsubsection*{Client}
\begin{verbatim}
USAGE: auth-client { -r | -l } username password
\end{verbatim}

Der Client wird entweder mit Option \texttt{-r} oder mit Option \texttt{-l}
ausgeführt. Diese Optionen entsprechen den folgenden Ausführungsmodi:

\begin{description}
  \item[Register] Mit der Option \texttt{-r} kann ein neuer Benutzer
    registriert werden. Falls der Benutzer (\texttt{username}) schon existiert,
    soll der Server dies dem Client entsprechend kommunizieren. Der Client soll
    in diesem Fall nach Ausgabe einer entsprechenden Fehlermeldung terminieren
    (\texttt{EXIT\_FAILURE}).  Existiert noch kein Benutzer mit dem angegebenen
    Namen (\texttt{username}) in der Datenbank des Servers, so soll der
    Benutzer angelegt werden und dies dem Client mitteilen. Der Client gibt
    eine entsprechende Erfolgsmeldung aus und terminiert
    (\texttt{EXIT\_SUCCESS}).
  \item[Login] Mit der Option \texttt{-l} kann sich ein
    bereits registrierter Benutzer einloggen. Der Server soll überprüfen, ob
    der gegebene Benutzer im System vorhanden ist oder nicht. Falls der
    Benutzer ungültig ist, soll der Server dies dem Client kommunizieren. Der
    Client soll nach Ausgabe einer entsprechenden Fehlermeldung terminieren
    (\texttt{EXIT\_FAILURE}).  Im Erfolgsfall soll der Client eine
    entsprechende Erfolgsmeldung ausgeben und auf weitere Befehle des Benutzers
    warten.
\end{description}

Nach einem erfolgreichen Login soll der Benutzer über die Standardeingabe
wiederholt einen der folgenden drei Befehle eingeben können:

\begin{description}
  \item[write secret] Der Benutzer soll eine Nachricht auf die Standardeingabe
    schreiben können, die als \texttt{secret} am Server gespeichert und mit dem
    eingeloggten Benutzer assoziiert wird.
  \item[read secret] Der Client soll das entsprechende Geheimnis
    ausgeben.
  \item[logout] Der Server loggt den Benutzer aus und der Client terminiert
    (\texttt{EXIT\_SUCCESS}).
\end{description}

Der Client erfragt den gewünschten Befehl bis sich der Benutzer ausloggt.
Gestalten Sie am Client ein einfaches Benutzerinterface, etwa:
\begin{verbatim}
Commands:
  1) write secret
  2) read secret
  3) logout
Please select a command (1-3):
\end{verbatim}

Die Inputs (Befehlsnummer und Geheimnis) sollen von der Standardeingabe
gelesen werden.

Um sicher zu gehen, dass der Client auch wirklich eingeloggt ist, soll bei
erfolgreichem Login eine zufällige Session-ID vom Server erzeugt, und an den
Client gesendet werden. Diese ID muss bei jeder Kommunikation vom Client an den
Server gesendet werden, um zu verifizieren, dass es sich um den
authentifizierten Client handelt.  Sollte ein Client mit falscher Session-ID
versuchen, zum Beispiel auf das \textit{secret} eines anderen Clients
zuzugreifen, so soll der Server eine Fehlermeldung ausgeben und dem Client
signalisieren, dass die Session ungültig ist.  Beim Logout soll die Session-ID
zerstört werden.


\subsection*{Datenformat}

\subsubsection*{Datenbank}

Die Datenbank soll als CSV-Datei gestaltet werden. Jede Zeile beinhaltet ein
Geheimnis \texttt{secret} eines Benutzers \texttt{username}, das mit einem
Passwort \texttt{password} geschützt ist.

\begin{verbatim}
username;password;secret
\end{verbatim}

Achten Sie darauf, keine unnötigen Leerzeichen zwischen den Semikolons
einzufügen, und insbesondere darauf, beim Speichern des secrets keinen
Zeilenumbruch (\verb|'\n'|) mitzuspeichern, wodurch leere Zeilen in der
Datenbank entstehen würden.

Sie dürfen für alle Felder (\texttt{username}, \texttt{password},
\texttt{secret}) eine maximale Länge definieren um Ihnen die Arbeit zu
erleichtern.

\subsubsection*{Beispiel}

\begin{verbatim}
Theodor;ilovemilka;I'm the best!
Anton;cforever;
Emil;osueisgreat;Something you would like to know, but I won't say ANYBODY!
\end{verbatim}

Der zweite Benutzer namens \texttt{Anton} hat in diesem Beispiel kein Geheimnis
gespeichert.


\osueguidelinesthree

\end{document}
